\documentclass[12pt,a4paper]{article}
\usepackage[T2A]{fontenc}
\usepackage[utf8]{inputenc}
\usepackage[english, russian]{babel}
\usepackage[OT1]{fontenc}
\usepackage{jupynotex}
\usepackage{amsmath, amsfonts, amssymb, multirow, textcomp, gensymb, wrapfig, csquotes}
\usepackage{graphicx}
% Set default figure placement to htbp
\def\fps@figure{htbp}
\makeatother
\setlength{\emergencystretch}{3em} % prevent overfull lines
\providecommand{\tightlist}{%
  \setlength{\itemsep}{0pt}\setlength{\parskip}{0pt}}
\usepackage{bookmark}
\IfFileExists{xurl.sty}{\usepackage{xurl}}{} % add URL line breaks if available
\urlstyle{same}
\hypersetup{
  hidelinks,
  pdfcreator={LaTeX via pandoc}}
\DeclareUnicodeCharacter{03BC}{μ}
\DeclareUnicodeCharacter{03BB}{λ}
\DeclareUnicodeCharacter{03C0}{π}
\DeclareUnicodeCharacter{2248}{≈}
\DeclareUnicodeCharacter{2212}{-}

\author{Демурин Артём Андреевич \\ }
\date{}

\begin{document}

\includegraphics[width=8.5in,height=11in]{media/cover.jpeg}

ISSN 1995-0780, Nanotechnologies in Russia, 2016, Vol. 11, Nos. 3--4,
pp. 144--152. © Pleiades Publishing, Ltd., 2016.

Original Russian Text © P.G. Rudakovskaya, D.N. Lebedev, M.V. Efremova,
E.K. Beloglazkina, N.V. Zyk, N.L. Klyachko, Yu.I. Golovin, A.G.
Savchenko, A.G. Mazhuga, 2016, published in Rossiiskie Nanotekhnologii,
2016, Vol. 11, Nos. 3--4.

Сore--Shell Magnetite--Gold Nanoparticles: Preparing and
Functionalization by Chymotrypsin P. G. Rudakovskaya \emph{a}, \emph{b},
D. N. Lebedev \emph{a}, M. V. Efremova \emph{a}, E. K. Beloglazkina
\emph{a}, \emph{b}, N. V. Zyk \emph{a}, N. L. Klyachko \emph{a},
\emph{b}, Yu. I. Golovin \emph{a}, \emph{c}, A. G. Savchenko \emph{b},
and A. G. Mazhuga \emph{a}, \emph{b}

aFaculty of Chemistry, Moscow State University, Moscow, 119991 Russia
bNational University of Science and Technology MISiS, Leninskii pr. 4,
Moscow, 119991 Russia cDerzhavin Tambov State University, ul.
Internatsional'naya 33, Tambov, 392000 Russia e-mail: polinaru@list.ru

Received October 12, 2015; in final form, December 14, 2015

\textbf{Abstract}---In this work we present the results of the synthesis
of magnetite--gold nanoparticles with a core--

shell structure. The preparation is carried out in several stages:
synthesis of the core, coating with a gold shell, purification of
magnetite--gold particles from an uncoated magnetite, and
functionalization of the surface with sulfur-containing ligands. The
conditions needed for the functionalization of nanoparticles with lipoic
acid and mercapto-methoxy polyethylene glycol are indicated in detail,
making it possible to determine the optimal conditions needed to achieve
an efficient purification and a maximum concentration of the particles
in a solution required for biological tests. The possibility of remotely
controlling the chymotrypsin properties using an alternating magnetic
field has been demonstrated by the example of magnetite--gold
nanoparticles.

The magnetite--gold nanoparticles which we have obtained are promising
for future biomedical applications.

\textbf{DOI:} 10.1134/S1995078016020166

INTRODUCTION

spectrum of sulfur-containing ligands, in turn simpli-fying the surface
modification {[}10{]}.

In the last decade, nanoparticles of iron oxides Actually, there are
numerous works devoted to the have become widespread in biomedical
studies. In a coating of MNPs with a gold shell {[}11--13{]};
neverthe-number of publications, magnetic nanoparticles are less, the
problem of the synthesis method optimiza-proposed to be used in the
treatment of oncological tion for such particles remains topical due to
the great pathologies using the hyperthermal method, trans-difference in
the nature of magnetite and gold surfaces port, and targeted drug
delivery to obtain tumor-selec-and, as a consequence, the necessity of a
purification tive MRT contrast agents {[}1--5{]}. Attempts to use
mag-stage. In connection with this, the main task of the netic
nanoparticles as agents--mediators for remotely study has become to
carry out the purification of mag-controlling the biochemical reactions
are known {[}6{]}.

netite--gold particles from the magnetite particles and Magnetic
nanoparticles are promising due to their conduct a complex
physicochemical study of the mag-high magnetic properties {[}7--9{]},
but they possess a netite--gold nanoparticles.

number of disadvantages: a susceptibility to rapid The next stage of the
work was the functionaliza-aggregation in a solution, difficulty of a
functionalization of the surface of magnetic nanoparticles with
sul-tion, and toxicity. A coating of magnetic nanoparticles
fur-containing organic ligands, a modification with with an inorganic
shell, e.g., gold, makes it possible to chymotrypsin being a ferment
whose catalytic proper-eliminate the disadvantages mentioned.

ties are studied in detail in the literature, and the sub-This work was
aimed at synthesizing and physico-sequent demonstration of the possible
application of chemically studying systems based on magnetite
magnetite--gold--protein particles in biomedicine.

nanoparticles (MNPs) with the preset properties for Chymotrypsin was
used as a model for studying a biomedical applications and studying the
possibility of mechanochemical action of an alternating magnetic the
effect of an alternating magnetic field on biomole-field, namely, the
possibility to adjust the catalytic cules immobilized on the surface of
the MNPs. In efficiency of the ferment immobilized on magnetic order to
eliminate the disadvantages listed, we used a nanoparticles at the
expense of varying its conforma-coating of MNPs with gold, which
stabilizes them, tion under the action of field forces.

reduces the toxicity of the particles to a minimum, and The novelty of
the work consists in the optimiza-makes it possible to modify
nanoparticles with a wide tion of the technique for preparing and
purifying mag-144

СORE--SHELL MAGNETITE--GOLD NANOPARTICLES

145

netite--gold nanoparticles and in the study of a non-column
chromatography with a filter on the bottom.

thermal mechanism of the magnetic field inf luence on Then the sorbent
was rinsed with 20 mL of 0.01 M

the kinetics of fermentative reactions with the partici-citrate buffer
(pH = 5) and a solution of Fe O @Au 3

4

pation of functionalized magnetic nanoparticles. A nanoparticles
stabilized by citrate ions (preliminarily regulation of catalytic
activity of a ferment immobi-stored in an ultrasonic bath for 20 min)
was deposited lized on magnetic nanoparticles using a field can
pres-onto the sorbent by drops of 15 mL. Then the sorbent ent a new
method for targeted drug delivery, while the was rinsed with 10 mL of
0.01 M citrate buffer (pH = 5).

application of magnetite--gold nanoparticles used as The resulting
sorbent with the Fe O @Au nanoparti-3

4

contrast agents {[}14{]} is a tool for the diagnostics. Taken cles
deposited onto it was suspended in 10 mL of DI together, this opens up a
way to new types of drugs---

H O; 10 mL of a solution of the ligands was dropped 2

theranostics.

into it with stirring, which was kept for a night. Then a dialysis of
the resulting solution was carried out in DI EXPERIMENTAL

water using SERVAPOR 44146 MWCO 12-14 kDa packages for dialysis (3 times
per 1 L for 3 days). Then \emph{General Information}

the solution was decanted and passed through a Milli-Transmission
electron microscopy is carried out on pore Hydrophilic PES 0.22 μm
injector filter. The a JEOL JEM-2100F/Cs/GIF (200 kV, 0.8 A) device.

resulting 18--20 mL of the nanoparticle solution was The data concerning
the sizes of nanoparticles were stored in a refrigerator.

obtained by a manual counting the nanoparticles in imageJ software. From
800 to 1000 particles were used for counting. The data of dynamic light
scattering \emph{Immobilization of Chymotrypsin on the Surface} (DLS)
were obtained on a ZetasizerNano ZS (173°) \emph{of Magnetite--Gold
Nanoparticles} device with a He-Ne laser (633 nm, 5 mV). An analysis of
nanoparticle trajectories was carried out on a Nano-An amount of 0--0.25
mL of the citrate buffer Sight NS500 device. Thermogravimetric analysis
(20 mM, pH 5.9) was added to 1 mL of the solution of (TGA) was carried
out on a Netzsch STA 449C deri-nanoparticles preliminarily stored in an
ultrasonic vatograph with a QMS 403C mass spectrometer. In bath for 20
min in such a way that the summary solu-order to carry out this
analysis, we used lyophilized tion volume would be equal to 1 mL. Then
the samples of nanoparticles prepared according to the required amount
of freshly prepared aqueous solutions above method. Heating of the
samples to 1000°C with of EDC and S-NHS was added to the solution (the a
rate of 5°C/min was carried out in Pt crucibles in air concentrations of
drain solutions were 10 mg/mL).

f lux. The method allows us to simultaneously register The optimal
conditions of the immobilization were a differential thermal analysis
(DTA) curve, a TGA 0.7 g of EDC, 0.2 g of S-NHS, 0.6 g of
alpha-chymo-curve, a differential thermogravimetric analysis trypsin,
and 250 μL of citrate buffer. After this, 200 μL

(DTGA) curve, and to determine the composition of of an aqueous solution
of alpha-chymotrypsin was volatile products using the mass spectra. The
experi-added to the solution (10 mg/mL). Then the resulting ments in a
magnetic field were carried out with the use solution was placed into a
shaker for 2 h at room tem-of an Astra-50 inductometer (OOO
Nanodiagnostika, perature. The last stage was purification from an
Russia) with a multilevel system of cooling and a sys-excess of an
unbound protein by a multiple centrifuga-tem for a cuvette thermostatic
control.

tion of the solution at 1000 g with the application of 100 kDa cutoff
centrifugal filters.

Synthesis of Magnetite Nanoparticles (9 ± 2 nm) in Diameter

\emph{Determining Alpha-Chymotrypsin Activity} The synthesis of MNPs was
carried out according to the method described in the literature
{[}15{]}.

The activity of alpha-chymotrypsin was determined using a SAAPPNA
substrate. First, 5 μL of a SAAPPNA substrate (the concentration is
0.0016 mM

\emph{Coating of Magnetite Nanoparticles with a Gold Shell} in a 1 : 1
mixture of dioxane--acetonitrile) was added The synthesis of
magnetite--gold nanoparticles was to 0.5 mL of a TRIS-HCl (pH 8.2)
buffer solution in a carried out according to {[}16, 17{]}.

quartz cuvette (the optical path length is 1 cm). Then the required
amount of a solution containing alpha-chymotrypsin was added to the
solution (to have a \emph{Purification {[}18{]} and Functionalization}
final activity of 0.0005--0.005 opt. un./min). After \emph{of
Magnetite--Gold Nanoparticles} intensive stirring, the cuvette was
placed into a spec-The amount of 0.15 g of Sephadex G-100 was
dis-trophotometer, in which the optical density was regis-solved in 10
mL of water and left for a night in a refrig-tered in time at a
wavelength of 380 nm. The ferment erator. The resulting solution was
poured into an activity was determined by the slope of the resulting
empty cartridge (12 g) for an Interchim PF-50SIHP

dependence.

NANOTECHNOLOGIES IN RUSSIA

Vol. 11

Nos. 3--4

2016

\includegraphics[width=3.68056in,height=2.33333in]{media/index-3_1.jpeg}

\includegraphics[width=3.47222in,height=2.30556in]{media/index-3_2.jpeg}

\includegraphics[width=2.66667in,height=1.98611in]{media/index-3_3.jpeg}

146

RUDAKOVSKAYA et al.

RESULTS AND DISCUSSION

The synthesis of Fe O @Au magnetite--gold 3

4

nanoparticles was performed in two successive stages.

At the first stage, according to the method of {[}15{]}, MNPs with an
average size of 9 ± 2 nm were obtained.

At the next stage {[}16, 17{]} the MNPs were coated with a gold shell by
reducing chlorauric acid with sodium citrate, which is both a reductant
and stabilizing ligand. From TEM data (Fig. 1) it can be seen that
magnetite--gold nanoparticles are obtained in a mixture with MNPs, which
is in agreement with the literature data {[}19--21{]}.

200 nm

Earlier, in order to purify Fe O @Au nanoparticles 3

4

from uncoated MNPs, the method of chromato-Fig. 1. TEM data for
magnetite--gold nanoparticles before graphic purification was proposed
{[}18{]}. The purifica-purification.

tion efficiency was estimated by transmission electron microscopy.
Figure 2 shows the magnetite--

\emph{Determining Protein Residual Activity} gold nanoparticles after
purification. The size of nanoparticles amounts to 23 ± 3 nm.

The required amount of nanoparticles with an The verification of the
purification efficiency was immobilized ferment in a citrate buffer
(20--200 μL, to also obtained from the electron spectroscopy data in
have a final activity of 0.001--0.005 opt. un./min) was UV and visible
spectral regions. Figure 3 shows the added into 1 mL of a TRIS-HCl
buffer (pH 8.2). Then optical spectra of Fe O nanoparticles before
coating, 3

4

the content was intensively stirred and distributed into unpurified
particles, and the nanoparticles after puri-two equal light-transparent
cuvettes. One of the fication and stabilization (sample 1). For the rest
of cuvettes was a control cuvette; 5 μL of a SAAPPNA the samples of
nanoparticles, the absorption spectrum substrate (a concentration of
0.0016 mM in 1 : 1 diox-after purification looks similar and has a
maximum at ane--acetonitrile mixture) was added to it, the content 531
nm. It can be seen from the optical spectra presented in Fig. 3 for the
solutions of Fe O nanoparti-was intensively stirred, and the cuvette was
placed into 3

4

a spectrophotometer. Simultaneously, another cuvette cles, and
unpurified and purified Fe O @Au nanopar-3

4

without a substrate was placed into a magnetic field.

ticles that the purification of nanoparticles causes a substantial
decrease in the absorption in a region of After 12 min exposure in a
field, 5 μL of a SAAPPNA

\textless460 nm. This is associated with the fact that, in the substrate
was added into this cuvette, the content was process of chromatography,
the uncoated MNPs are again intensively stirred, and the cuvette was
placed removed.

into a spectrophotometer.

The stage of Fe O @Au nanoparticle surface func-3

4

The inf luence of the field was estimated as the ratio tionalization is
conjugated with the purification stage.

of the activity in the cuvette subjected to the field to In order to
stabilize and functionalize Fe O @Au 3

4

that in the control cuvette.

nanoparticles, we used in this work two different Fraction of particles

16

14

12

10

8

6

4

2

500 nm

0

10

15

20

25

30

35

Diameter of particles, nm

\textbf{Fig. 2.} TEM data for magnetite--gold nanoparticles after
purification. Hystogram of size distribution.

NANOTECHNOLOGIES IN RUSSIA

Vol. 11

Nos. 3--4

2016

\includegraphics[width=3.94444in,height=1.51389in]{media/index-4_1.jpeg}

СORE--SHELL MAGNETITE--GOLD NANOPARTICLES

147

Optical density, rel. units

λ

λ

0.25

max = 518 nm

max = 531 nm

Unpurified

0.23

nanoparticles

Fe

0.21

Fe

3O4@Au@PEG

3O4@Au

0.19

0.17

0.15

0.13

0.11

0.09

Fe3O4

0.07

0.05

400

420

440

460

480

500

520

540

560

580

λ, nm

\textbf{Fig. 3.} Spectroscopic data in a visible region for MNPs (light
gray), magnetite--gold nanoparticles before purification (black), and
magnetite--gold nanoparticles after purification (gray).

ligands: lipoic acid (thioctic acid C H O S ) and stage: 10\%. The table
contains information about the 8

14

2 2

polyethylene glycol (PEG) with a terminal thio group samples of
nanoparticles obtained in this work.

(HS--(CH --CH --O--) --CH , 5000 g/mol, \emph{n} ≈ 113).

2

2

n

3

An analysis of colloidal solutions of nanoparticles The structure of the
ligands is shown in Fig. 4. Thio-was carried out using nanoparticle
tracking analysis polyethylene glycol was chosen in order to impart col-

(NTA) and dynamic light scattering (DLS) methods.

loid stability for the solutions of nanoparticles; many An analysis of
the trajectories of nanoparticles allows authors carry out a
modification of the nanoparticle us to obtain information about the
sample concentra-surface using similar polymer molecules {[}22{]}. In
order tion and particle size distribution.

to immobilize molecules to the surface of nanoparti-The colloidal
solutions of nanoparticles we have cles, it is necessary to use the
ligands containing func-obtained are stable in time, so after 2 months
of storing tional groups, e.g., carboxyl or amine groups; lipoic at +4

acid suits this criterion.

°C the optical density of the solutions in the absorption maximum (531
nm) amounts to 0.98--1.02

In this work we proposed using two ligands in vari-of the initial one.
The concentrations of nanoparticles ous proportions to achieve, on the
one hand, colloid in colloidal solutions obtained by an analysis of the
stability and, on the other hand, the possibility of trajectories of
nanoparticles (table) correlate with the immobilization and studying the
effect of the mag-solution optical density and the amount of PEG added
netic field on the ferment immobilized to the surface at the stage of
the nanoparticle surface modification of nanoparticles.

(Fig. 5).

In order to determine the number of the ligands An increase in the
amount of HSPEGOCH leads 3

needed for a modification of the surface of nanoparti-to an increase in
the concentration of nanoparticles in cles, we have carried out a
preliminary estimation of the solution. This is caused by the fact that
PEG facil-the maximal number of ligands that can be accommo-itates the
removal of nanoparticles from Sephadex dated on the nanoparticle
surface. This estimation was G100 at the purification stage. Probably
just such a carried out based on the nanoparticle surface area and
number of nanoparticles transfer from a carrier into typical areas
occupied by sulfur-containing ligands on the solution, which is
colloidal stable. Therefore, an a gold surface. Thus, the amount of
ligand moles to be admixture of PEG makes it possible to increase the
added into 10 mL of DI water at the last stage of syn-concentration of
nanoparticles in a colloidal solution.

thesis for the complete coating of the nanoparticle sur-It is worth
noting that the sizes (diameters) of face was estimated to be 1.03 ×
10--6 mole. This nanoparticles indicated in the table are hydrodynamic.

amount is conveniently expressed as a percentage of In addition, the
sizes indicated in the table are the the full gold moles added at the
particle purification weighted average values. Thus, in the case of
samples 1--4, the average size calculated by the NTA method is in a
range of 37--42 nm; the real maximum O

of the particle size distribution peak is in a range of 31--34 nm. These
values are quite expected in the case O

CH

OH

3

of particles 23 nm in diameter coated with PEG

HS

O

n

(5000 g/mol) and surrounded by a hydrate shell.

S S

A decrease in the PEG amount leads to a broadening of the particle size
distribution. It is worth noting that the data obtained by different
methods agree with each \textbf{Fig. 4.} Ligands used in this work:
lipoic acid (on the left) and HSPEGOCH

other. A small difference (on average, a greater size 3 (on the right).

NANOTECHNOLOGIES IN RUSSIA

Vol. 11

Nos. 3--4

2016

\includegraphics[width=5.44444in,height=2.56944in]{media/index-5_1.jpeg}

148

RUDAKOVSKAYA et al.

Data on all the samples: ligand content, size, and concentration
\emph{D} (according to \emph{D} (according to Concentration

Optical density

No. \% of lipoic acid \% of PEG (5000) NTA data)

DLS data)

(according to NTA data) in a peak at λ 531 nm 1

0

10

48

60

3.7 × 1011

0.814

2

10

10

49

62

3.0 × 1011

0.732

3

1

1/2

45

64

2.2 × 1011

0.667

4

1

1/4

50

68

1.2 × 1011

0.348

5

1

1/8

66

65

8.4 × 1010

0.308

6

1

1/16

59

101

3.0 × 1010

0.122

7

10

0

93

109

1.5 × 1010

0.066

according to DLS measurements) can be accounted a sample practically
does not vary. A small increase in for by the fact that the methods are
based on different the mass after 400°C was detected in the case of
sam-physical phenomena.

ple 2. A peak at 280°C is clearly seen on DTGA curves (the intensity in
the maximum is --2\%/min) for sam-In order to estimate the number of
ligands located ple 1 and two peaks at 245°C and at 265--270°C

at the surface of nanoparticles, we have carried out a (intensities in
the maximum are --6\%/min and thermogravimetric analysis of samples 1
and 2. The

--1.5\%/min, respectively) for sample 2. For both sam-choice of just
these samples for analysis is caused by ples in a region of 235--320°C,
peaks on the curves of the possibility to use the data we have obtained
for an the current of an ion with \emph{m}/ \emph{z} = 44 g/mol (CO )
and estimation of a maximum PEG amount that can be 2

on the DSC curves are observed.

located on a nanoparticle, verifying the presence of lipoic acid on the
surface of nanoparticles, and an estiOn the basis of literature data
{[}23{]}, we can con-mation of its amount. The curves of TGA, DTGA,
clude that the initial small loss in the mass by the sam-differential
scanning calorimetry (DSC), and a cur-ples is associated with a loss of
water, while a substan-rent of an ion with m/z = 44 g/mol (CO ) are
pre-tial mass loss at 235--320

2

°C corresponds to the

sented in Fig. 6. The general shape of the TGA curve decomposition of
the ligands bound to the surface of is similar for both samples: first,
a small loss in the nanoparticles. This is also proven by an increase in
an sample mass occurs (4.3\% for sample 1 and 6.3\% for ion current of
CO in this temperature region. On the 2

sample 2 up to 200°C); then at 235°C a rather sharp basis of sample mass
losses in this temperature region, loss in the mass starts (17.5\% for
sample 1 and 19.7\%

it is possible to evaluate the number of ligands bound for sample 2 up
to 320°C), and after 320°C the mass of to nanoparticles.

Concentration, part./mL

3.5E+11

3.0E+11

2.5E+11

2.0E+11

according to NTA data

1.5E+11

according to opt. abs.

1.0E+11

5.0E+10

0

0.2

0.4

0.6

0.8

1.0

PEG, \% of Au mole

\textbf{Fig. 5.} Concentrations of nanoparticles in colloidal solutions
obtained by the NTA method and calculated from the optical density of
the solutions.

NANOTECHNOLOGIES IN RUSSIA

Vol. 11

Nos. 3--4

2016

\includegraphics[width=4.54167in,height=3.22222in]{media/index-6_1.jpeg}

\includegraphics[width=4.56944in,height=3.20833in]{media/index-6_2.jpeg}

СORE--SHELL MAGNETITE--GOLD NANOPARTICLES

149

Ionic current 1011/A

π/\%

DSC, mW/mg DTGA

1

ecc

10

Variation in the mass 4.31\%

15

3

4

Variation in the mass 17.47\%

9.0

Residual mass 78.98\% (999.7°C)

3

Variation in the mass 1.64\%

1

80

10

8.0 2

Variation in the mass 2.38\%

1

60

Peak at 289.3°C

7.0

2

\emph{2} 5

0

6.0

40

−1

3

0

−2

5.0

20

4

\emph{m}/ \emph{z} 44

4

−3

−5

4.0 −4

0

100

200

300

400

500

600

700

800

900

Temperature, °C

Ionic current 1011/A

π/\%

DSC, mW/mg DTGA

Peak at 246.5°C

120

ecc

7.2

2

\emph{1} Peak at 148.4°C

3

100

Variation in the mass 6.33\%

5

3

7.0

0

4

Variation in the mass 19.73\%Residual mass 83.69\% (999.6°C) \emph{1}

80

\emph{m}/ \emph{z} 44

−

Variation in the mass 9.50\%

2 6.8

2

2

0

−4

60

6.6

−6

6.4

40

−5

−8

6.2

20

−10

4

Peak at 806.0°C

6.0 −10

−

0

12

100

200

300

400

500

600

700

800

900

Temperature, °C

\textbf{Fig. 6.} Results of thermogravimetric analysis for samples 1
(above) and 2 (below). Mass losses by sample: TGA curve ( \emph{1}), its
deriv-ative ( \emph{2}), DSC curve ( \emph{3}), and curve of a current
of an ion with \emph{m}/ \emph{z} = 44 g/mol (CO2) ( \emph{4}).

In the case of fully PEG-liganded sample 1, a loss row and more
intensive peak at 245°C can be associ-of 17.5\% of mass occurs. A
recalculation of this value ated with the process of lipoic acid
decomposition.

into percentages of gold moles taken for a coating of A summary mass
loss due to these processes amounts 15 mL of MNPs by a shell gives 0.9 ±
0.1\%. In the case to 19.7\%. A profile analysis of the peaks makes it
pos-of sample 2, as was mentioned above, two peaks can be sible to
separate the contributions of both processes separated on DTGA curve. A
broad weak peak at 265--

and, on the basis of peak areas, infer that a loss of 270°C can be
attributed to the process of PEG decom-12.7 ± 1\% of mass is associated
with the PEG decomposition (like in the case of a previous sample); a
nar-position, while the lipoic acid decomposition is NANOTECHNOLOGIES IN
RUSSIA

Vol. 11

Nos. 3--4

2016

\includegraphics[width=2.95833in,height=2.63889in]{media/index-7_1.png}

\includegraphics[width=2.93056in,height=2.61111in]{media/index-7_2.png}

150

RUDAKOVSKAYA et al.

lization of alpha-chymotrypsin. Alpha-chymotrypsin

\%

was chosen as a model ferment, because now a great 16

amount of information about this protein exists in the 14

literature: construction (crystalline structure), catalytic properties,
model substrates, etc. {[}24{]}. The activ-12

ity of chymotrypsin can be rather easily measured using elaborated
methods and various substrates. In 10

the course of this work, an immobilization of alpha-8

chymotrypsin was carried out on the surface of nanoparticles. In order
to carry out the process of 6

crosslinking, we used a carbodiimide method. EDC

and S-NHS are used in the carbodiimide method as 4

the main linking reagents; the method employs car-2

boxyl groups of nanoparticles and amines of protein.

Thus, in this work we have obtained the Fe O @Au--

3

4

0

alpha-chymotrypsin systems (sample 8) based on 0

50 100 150 200 250 300 350

magnetite--gold nanoparticles functionalized with \emph{d}, nm

lipoic acid and PEG with 1/16 ratio (sample 6), since

\%

this sample is optimal for further studies. In order to 5

avoid the system aggregation, it is possible either to reduce the amount
of linking reagent or increase the volume of an added citrate buffer. It
should be noted 4

that the activity of the systems correlates with the concentrations of
the precursors taken at an immobilization stage. The activity falls with
a decrease in amounts 3

of ferment, crosslinking, and citrate.

The basic tool allowing us to study the colloidal 2

solutions of nanoparticles obtained after protein immobilization is the
NTA method. As was mentioned above, this method provides information on
the 1

concentration of nanoparticles in a solution and on the size
distribution of nanoparticles. For sample 8, an 0

effect of broadening of the size distribution of particles 0

50 100 150 200 250 300 350

is detected; the average diameter increases by more \emph{d}, nm

than two times. Figure 7 shows the size distributions of particles for
samples 6 (prior to protein crosslinking) \textbf{Fig. 7.} Size
distributions of the particles according to NTA and 8 (after protein
crosslinking).

data for samples 6 (before linking) and 8 (after linking).

In order to reveal the effect of the magnetic field on the activity of a
ferment immobilized to the surface of nanoparticles, we studied in this
work the kinetics of responsible for 7 ± 1\% loss of mass. A
recalculation of SAAPPNA substrate hydrolysis after the exposure of
these values into percentages of gold moles gives 1.1 ±

the system to an alternating magnetic field (110 kA/m, 0.1\% for PEG and
5.6 ± 2\% for lipoic acid.

50 Hz); a sample that was not exposed to magnetic Thus, the TGA data
indicate that a maximum field was used as a control. For sample 8 and
the amount of PEG that can be bound to the surface of SAAPPNA substrate,
residual activity after the action nanoparticles is 1\%. The binding of
this amount of of the magnetic field amounted to 62 ± 10\% of the
ini-PEG does not prevent a binding of lipoic acid; its tial one. Figure
8 shows a kinetic experiment for this amount can achieve 5\% of gold
moles.

sample.

For magnetite--gold--chymotrypsin samples obtained Thus, in this work we
have studied the functional-on the basis of 3--5 samples, no effect of
the magnetic ization of magnetite--gold nanoparticles at a
purifica-field on the activity of an immobilized protein was tion stage
and optimal concentrations of the ligands revealed, which is associated
with the large PEG

are revealed in order to achieve colloidal stability and amount in the
system.

preserve functional properties.

The described decrease in the fermentative activity After a
comprehensive characterization of the after the action of the magnetic
field can be explained nanoparticles, in this work we carried out an
immobi-using the concept of nanomechanical action (Fig. 9),
NANOTECHNOLOGIES IN RUSSIA

Vol. 11

Nos. 3--4

2016

\includegraphics[width=6.5in,height=2.77778in]{media/index-8_1.jpeg}

СORE--SHELL MAGNETITE--GOLD NANOPARTICLES

151

Optical density, rel. units

0.100

0.095

\emph{y} = 0.00209 \emph{x} + 0.06914

0.090

\emph{R} 2 = 0.99577

\emph{y} = 0.00123 \emph{x} + 0.05545

\emph{R} 2 = 0.98772

0.085

0.080

0.075

62 + 10\%

0.070

0.065

0

2

4

6

8

10

12

14

16

18

20

Time, min

\textbf{Fig. 8.} Inf luence of the magnetic field on the activity of
immobilized ferment for sample 8.

which was proposed by us earlier {[}25{]}. In the case of CONCLUSIONS

this mechanism, the ferment should be simultane-In the course of this
work, core--shell magnetite--

ously attached (``relinked'') to several different gold nanoparticles
with a size of 23 ± 3 nm were syn-nanoparticles. Then, in agreement with
the literature, thesized and purified. The nanoparticles were
func-forces of hundreds of pN can be developed, which are tionalized
with bifunctional lipoic acid due to the sufficient for a change in the
conformation and, covalent binding of sulfur to the gold surface. The
hence, in the activity of an immobilized ferment.

inf luence of the lipoid acid/PEG ratio on the colloidal stability of
the nanoparticles that we obtained was Thus, in this work we have
demonstrated the pos-studied; it was found that the optimal lipoic
acid/PEG

sibility of remotely controlling biochemical processes ratio is 1 : 16.
The magnetite--gold nanoparticles were due to the nanomechanical action
on the ferments covalently functionalized with a model chymotrypsin
immobilized to magnetic nanoparticles. The use of ferment and the
possibility of remotely controlling the such an impact mechanism opens
up prospects for operation of the ferment immobilized to magnetic
various medical applications.

nanoparticles using an alternating low-frequency magnetic field was
demonstrated. The nanoparticles which we have obtained are promising for
biomedicine and pharmacology.

B

Au

ACKNOWLEDGMENTS

Au

Fe O

3

4

Fe O

This work was carried out with financial support \textbf{3}

4

from the Russian Foundation for Basic Research, \emph{B}

grant no. 14-13-00731 (studies), and the Ministry of Education and
Science of the Russian Federation, project no. 14.604.21.0007 (synthesis
of nanomateri-Fe O

3

4

als), unique identifier of applied scientific research (of \textbf{Fe O}

3

4

the project) RFMIFI60414X0007.

Au

Au

REFERENCES

1. R. Hergt, S. Dutz, R. Müller, et al., ``Magnetic particle
\textbf{Fig. 9.} Schematic of the concept of a remote
nanomechan-hyperthermia: nanoparticle magnetism and materials ical
impact by an external magnetic field on immobilized development for
cancer therapy,'' J. Phys.: Condens.

biomolecules.

Matter \textbf{18}, 2919--2934 (2006).

NANOTECHNOLOGIES IN RUSSIA

Vol. 11

Nos. 3--4

2016

152

RUDAKOVSKAYA et al.

2. K. J. Widder, A. E. Senyei, and D. G. Scarpelli, ``Mag-16. P. G.
Rudakovskaya, E. K. Beloglazkina, A. G. Majouga, netic microspheres:a
model system for site specific drug and N. V. Zyk, ``Synthesis and
characterization of ter-delivery in vivo,'' Proc. Soc. Exp. Biol. Med.
\textbf{58}, 141--

pyridine-type ligand-protected gold-coated Fe3O4

146 (1978).

nanoparticles,'' Mendeleev Commun. \textbf{20}, 158--160

3. Y. J. Wang, ``Superparamagnetic iron oxide based MRI (2010).

contrast agents: Current status of clinical application,''

17. A. Majouga, M. Sokolsky-Papkov, A. Kuznetsov, Quant. Imag. Med.
Surg. \textbf{1}, 36--40 (2011).

D. Lebedev, M. Efremova, E. Beloglazkina, P. Ruda-4. C. Xu and S. Sun,
``New forms of superparamagnetic kovskaya, M. Veselov, N. Zyk, Y.
Golovin, N. Klyac-nanoparticles for biomedical applications,'' Adv. Drug
hko, and A. Kabanov, ``Enzyme-functionalized gold-Deliv. Rev.
\textbf{65}, 732--743 (2013).

coated magnetite nanoparticles as novel hybrid nano-materials:
synthesis, purification and control of enzyme 5. F. Hasany, H.
Abdurahman, R. Sunarti, and R. Jose, function by low-frequency magnetic
field,'' Colloids

``Magnetic iron oxide nanoparticles: chemical synthesis Surf. B:
Biointerfaces \textbf{125}, 104--109 (2015).

and applications review,'' Curr. Nanosci. \textbf{9}, 561--575

(2013).

18. N. L. Klyachko, M. Sokolsky-Papkov, N. Pothayee, 6. A. M. Derfus, G.
Maltzahn, T. J. Harris, et al., M. V. Efremova, D. A. Gulin, A. A.
Kuznetsov,

``Remotely triggered release from magnetic nanoparti-A. G. Majouga, J.
S. Riff le, Y. I. Golovin, and cles,'' Adv. Mater. \textbf{19},
3932--3936 (2007).

A. V. Kabanov, ``Changing the enzyme reaction rate in magnetic
nanosuspensions by a non-heating magnetic 7. J. Chomoucka, J.
Drbohlavova, D. Huska, V. Adamb, field,'' Angew. Chem. Int. Ed. Engl.
\textbf{51}, 12016 (2012).

and J. Hubalek, ``Magnetic nanoparticles and targeted drug delivering,''
Pharm. Res. \textbf{62}, 144--149 (2010).

19. T. Ahmad, H. Bae, I. Rhee, et al., ``Gold-coated iron oxide
nanoparticles as a T2 contrast agent in magnetic 8. T. K. Nguyen Thanh,
\emph{Magnetic Nanoparticles: From} resonance imaging,'' J. Nanosci.
Nanotechnol. \textbf{12}, \emph{Fabrication to Clinical Applications}
(CRC, Taylor and 5132--5137 (2012).

Francis, Boca Raton, London, New York, 2012).

20. Yu-Hui Bai, Jin-Yi Li, Jing-Juan Xu, and Hong-Yuan 9. P. Majewski
and B. Thierry, ``Functionalized magne-Chena, ``Ultrasensitive
electrochemical detection of tite nanoparticles---synthesis, properties,
and bio-DNA hybridization using Au/Fe

applications,'' Crit. Rev. Solid State Mater. Sci. \textbf{32}, 3O4
magnetic composites combined with silver enhancement,'' Affiliat.
Ana-203--215 (2007).

lyst \textbf{135}, 1672--1679 (2010).

10. M. C. Daniel and D. Astruc, ``Gold nanoparticles: 21. I. Robinson,
D. Tung, S. Maenosono, et al., ``Synthesis assembly, supramolecular
chemistry, quantum-size-of core-shell gold coated magnetic nanoparticles
and related properties, and applications toward biology, their
interaction with thiolated DNA,'' Nanoscale \textbf{12}, catalysis, and
nanotechnology,'' Chem. Rev. \textbf{104}, 293--

2624--2630 (2010).

346 (2004).

22. W. Qian, M. Murakami, Y. Ichikawa, and Y. Che, 11. P. G.
Rudakovskaya, E. K. Beloglazkina, A. G. Mazhuga,

``Highly efficient and controllable PEGylation of gold N. L. Klyachko,
A. V. Kabanov, N. V. Zyk, ``Synthesis nanoparticles prepared by
femtosecond laser ablation in of magnetite-gold nanoparticles with
core-shell struc-water,'' J. Phys. Chem. C \textbf{115}, 23293--23298
(2011).

ture,'' Vestn. Mosk. Univ., Ser. Khim. \textbf{56}, 181--189

(2001).

23. J. Vidal-Vidal, J. Rivasb, M. A. López-Quintela, ``Synthesis of
monodisperse maghemite nanoparticles by the 12. X. Zhao, Y. Cai, T.
Wang, et al., ``Preparation of alkan-microemulsion method,''
ColloidSurf. A: Physico-ethiolate-functionalized core/shell Fe3O4--Au
nanoparti-chem. Eng. Asp. \textbf{288}, 44--51 (2006).

cles and its interaction with several typical target molecules,'' Anal.
Chem. \textbf{80}, 9091--9096 (2008).

24. W. Appel, ``Chymotrypsin: molecular and catalytic 13. U. Tamer, Y.
Gündoğdu, I. HakkıBoyacı, et al., ``Syn-properties,'' Clin. Biochem.
\textbf{19}, 317--322 (1986).

thesis of magnetic core--shell Fe

25. Yu. I. Golovin, N. L. Klyachko, D. Yu. Golovin, 3O4--Au nanoparticle

for biomolecule immobilization and detection,''

M. V. Efremova, A. A. Samodurov, M. Sokolski-Pap-J. Nanopart. Res.
\textbf{12}, 1187--1196 (2009).

kov, and A. V. Kabanov, ``A new approach to the con-14. T. Ahmad, H.
Bae, I. Rhee, et al., ``Gold-coated iron trol of biochemical reactions
in a magnetic nanosus-oxide nanoparticles as a T2 contrast agent in
magnetic pension using a low-frequency magnetic field,'' Tech.

resonance imaging,'' J. Nanosci. Nanotechnol. \textbf{12}, Phys. Lett.
\textbf{39}, 240 (2013).

5132--5137 (2012).

26. J. C. Phillips, K. Schulten, et al., ``Scalable molecular 15. Ch. K.
Lo, D. Xiao, and M. F. Choi, ``Homocysteine-dynamics with NAMD,'' J.
Comput. Chem. \textbf{26}, 1781--

protected gold-coated magnetic nanoparticles: synthe-1802 (2005).

sis and characterization,'' J. Mater. Chem. \textbf{17}, 2418--

2427 (2007).

Translated by R. Litvinov

NANOTECHNOLOGIES IN RUSSIA

Vol. 11

Nos. 3--4

2016

\end{document}
